\documentclass[12pt]{article}
\usepackage[utf8]{inputenc}
\usepackage[russian]{babel}
\begin{document}
\title{Сравнительный анализ подходов к созданию электронных филологических коллекций}
\author{Андреев А. В., Бухаркин П. Е., Пономарева М. В.}
\maketitle

В отличие от чисто лингвистических корпусов текстов, принципы построения которых достаточно хорошо
изучены, методологии разработки электронных коллекций, ориентированных на
историко-литературоведческий анализ, фактически не существует. 

В настоящее время на материале русской литературы развиваются несколько электронных коллекций, из
которых аиболее известны, вероятно, проекты <<ФЭБ-Веб>>, <<Российская виртуальная библиотека>>, 
электронная библиотека <<ImWerden>>\footnote{Международный проект <<Гутенберг>>,
одна из наиболее известных и полных электронных библиотек, текстов на русском языке практически не
содержит}. В данном работе мы не будем касаться вопросов, связанных с комплектацией фондов электронных
коллекций, а также эдиционных принципов, лежащих в основе этих коллекций; мы ограничимся лишь проблемами
организации текстов внутри коллекции.

Большинство электронных коллекций, включая и вышеперечисленные, в целом следуют традиционному <<библиографическому>> 
принципу, при котором тексты в коллекции структурируются посредством нескольких каталогов,
в первую очередь. Так, в <<Русской виртуальной библиотеке>> тексты организованы в 4-уровневую иерархию
(раздел -- автор -- издание -- конкретное произведение); в библиотеке <<ImWerden>> издания упорядочены с
помощью алфавитного (автор/издание\footnote{Библиотека <<ImWerden>> оперирует не отдельными произведениями,
а изданиями целиком, оцифрованными в формате PDF}) и неформального систематического каталогов. Ни одна из этих коллекций
не имеет средств автоматизированного библиографического поиска -- пользователь просто имеет дело с упорядоченными
списками. Проект <<ФЭБ-Веб>> имеет несколько более сложную структуру: тексты распределены по нескольким \emph{электронным
научным изданиям (ЭНИ)}, внутри которых собраны издания, относящиеся к одному автору или к одной теме; структура каждого издания
вопроизводит структуру оригинала, с разбиением на тома, главы и т.~п. Помимо этого, <<ФЭБ-Веб>> представляет сводный
алфавитный каталог и средства для библиографического поиска по названию произведения, автору и дате публикации (ценность
которых, к сожалению, значительно снижается из-за поиска только в пределах одного ЭНИ, но не во всей коллекции).

Организация текстов в виде каталогов ориентирована только на внешний поиск текста (т.~е. на поиск по тому или иному
элементу метаданных --- библиографического описания). Наряду с этим, все рассмотренные коллекции предоставляют в том или
ином виде и средства \emph{внутреннего} поиска, т.~е. поиска по содержимому текста. В первую очередь в этой роли выступают
механизмы полнотекстового поиска. Полнотекстовый поиск --- это мощный инструмент анализа, применение которого стало в полной
мере возможным только после перехода к цифровой форме представления текстов; однако применительно к задачам филологического
исследования художественной литературы с его использованием связаны две существенные проблемы:
\begin{enumerate}
\item слова, не входящие в основной фонд русского языка, как-то: имена собственные, окказионализмы, диалектизмы и т.~п.,
  случаи употребления которых как раз представляют особый интерес для филолога, могут быть лемматизированы с большим
  трудом
\item в коллекции могут содержаться и тексты, грамматика которых отличается от современной нормативной (например,
  произведения русской литературы XVIII~в.)
\end{enumerate}

К сожалению, во всех трех рассмотренных коллекциях эти проблемы полностью вытесняются гораздо более серьезной проблемой,
а именно: выбором в качестве инструмента поиска стандартных средств Google (ImWerden) и Yandex (РВБ и ФЭБ-Веб).
Поскольку алгоритмы индексации и поиска, используемые в этих средствах, являются коммерческой тайной соответствующих
компании, результаты поиска принципиально не верифицируемы и, следовательно, \emph{категорически} не могут быть использованы
как основа для серьезного научного исследования.

Помимо такого общего поиска, РВБ и ФЭБ-Веб также содержат научно выверенные конкордансы к отдельным произведениям и
авторам; также из средств внутреннего поиска в РВБ присутствуют метрико-строфические указатели к отдельным авторам.

Следует отметить, что большинство электронных коллекций являются достаточно закрытыми --- не в плане ограничения доступа
к информации, а в плане возможности интеграции со сторонними инструментами анализа и обработки данных. Ни одна из рассмотренных
коллекций не предоставляет никаких программных интерфейсов (например, Web-сервисов) для работы с текстами коллекции. Более того,
ни одна из этих коллекций не предоставляет доступа к машинно-читаемым версиям текстов\footnote{Для проектов ФЭБ-Веб и РВБ существование
таких версий заявлено, однако пользователю они не доступны; более того, только для РВБ существует описание формата этих
внутренних версий.}   и библиографических описаний.
Таким образом, пользователь имеет дело исключительно с отформатированными HTML-страницами и со встроенными средствами
поиска и навигации, которые, как было выше указано, имеют достаточно ограниченную функциональность.

\end{document}